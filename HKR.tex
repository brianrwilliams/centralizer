Now we swiftly describe how to use methods like those in this paper to prove a version of this powerful result.

\owen{I've gotten myself pretty mixed up about this \dots too many duals with infinite dimensional vector spaces, and issues with whether you can base-change a semi-free algebra.}

First, we wish to define ${\rm Poly}_n(A)$ for an arbitrary dg commutative algebra $A$ in a way that gives the ``correct'' derived answer, i.e., the answer up to quasi-isomorphism of dg commutative algebras.
Concretely, one should replace $A$ by a quasi-isomorphic but better behaved algebra $\widetilde{A}$, 
such as semi-free resolution or cofibrant replacement with respect to some nice model category structure.
Then one computes the tangent complex $\TT_{\widetilde{A}}$, i.e., the derived derivations.
The {\em shifted polyvector fields} of $A$ means
\[
{\rm Poly}_n(A) \simeq \Sym_{\widetilde{A}} (\TT_{\widetilde{A}}[-n]),
\]
which is again a dg commutative algebra up to quasi-isomorphism.

%Second, we introduce a technical hypothesis  needed to apply our methods.
%We need an {\em augmented} dg commutative algebra $\epsilon: A \to k$ that admits a semifree resolution of the form $(\Sym(V^\vee), \partial)$, where $V$ is some graded vector space.
%In other words, we need that the generators can be realized as the graded linear dual of some other graded vector space.
%We say such an algebra $A$ is of {\em type~($\star$)}.

\begin{prp}
Let $A$ be an augmented dg commutative algebra.
Then 
\[
{\frak Z}_n(A) \simeq {\rm Poly}_n(A).
\]
\end{prp}

\owen{This is really turning into a mess.}

\begin{proof}
We are free to replace $A$ by a semifree resolution $(\Sym(V), \partial)$,
where $\partial$ sends $V$ to the augmentation ideal $\Sym^{> 0}(V)$.
Below we will (abusively) use $A$ to denote this semifree resolution,
as the original $A$ is irrelevant for us.
Ignoring the differential, we see that our situation resembles our central result for an abelian Lie algebra,
and so we now mimic the proof of our central result.

There is a natural homotopy-locally constant sheaf of cochain complexes $\Omega^\bu$ with differential $\d_{dR}$,
which is a soft replacement for the locally constant sheaf $\RR$.
Taking duals, we get a cosheaf.
In the same way, we define the factorization algebra 
\[
F_A = (\Sym((\Omega^\bu)^* \otimes V ), \d_{dR} + \partial) 
\]
which is manifestly commutative and is quasi-isomorphic to $A$ on every disk.
Hence $F_A$ is a commutative, locally constant factorization algebra that models~$A$.

Similar techniques allow us to construct a factorization algebra modeling the polyvector fields ${\rm Poly}_n(A)$.
Define
\[
Z_A = (\Hom^\bu(\Sym((\Omega^\bu)^* \otimes V), \Sym(\Omega_c^* \otimes V), [\d_{dR} + \partial,-]).
\]
Note its similarity to our model $\bZ_\fg$ for the center of~$\bU_n \fg$.
%There is a canonical shifted Poisson structure here by extending the canonical  pairing
%\[
%(\Omega^\bu \otimes V)^* \otimes \Omega^\bu_c \otimes V \to \RR
%\]
%arising from the inclusion $\Omega^\bu_c \otimes V \hookrightarrow \Omega^\bu \otimes V$.

Let us verify that 
\[
Z_A(D) \simeq {\rm Poly}_n(A)
\]
for any open disk $D$.
Due to the Poincar\'e lemma, we have canonical quasi-isomorphisms $(\Omega^\bu(D))^* \otimes V \xto{\simeq} V$ and $\Omega^\bu_c(D) \otimes V \xto{\simeq} V[-n]$,
where we view $V$ as simply a graded vector space but the de Rham complexes have the differential $\d_{dR}$.
(The differential $\partial$ will make its appearance shortly.)
Thus we have quasi-isomorphisms
\[
i: \Sym((\Omega^\bu(D))^* \otimes V) \xto{\simeq} \Sym(V)
\]
and 
\[
i_c: \Sym(\Omega_c^*(D) \otimes V) \xto{\simeq} \Sym(V[-n]).
\]
As $\partial$ acts only on the $V$-factors and not the de Rham factors,
we can modify all the complexes by adding $\partial$ to the differential.
Finally, if we fix a quasi-isomorphism 
\[
s: \Sym(V) \xto{\simeq} \Sym((\Omega^\bu(D))^* \otimes V),
\]
we get a map
\[
\begin{array}{ccc}
Z_A(D) & \to & \Hom^\bu(A,A)\\
\Phi & \mapsto & i_c \circ \Phi \circ s
\end{array}
\]
that is a quasi-isomorphism.
(Note that $s$ depends on the disk~$D$. 
We do not construct a map that intertwines with any structure maps of~$Z_A$.)

We now turn to constructing a map from this factorization algebra $Z_A$ to the center ${\frak Z}_n(A)$.
The key is to show that $Z_A$ fits into a commuting triangle as in Definition~\ref{D:centralizer},
because then we know $Z_A$ has a canonical map to~${\frak Z}_n(A)$.
This action is quite easy to see here: there is a canonical map
\[
\Hom^\bu(\Sym((\Omega^\bu)^* \otimes V), \Sym(\Omega_c^* \otimes V) \otimes \Sym((\Omega^\bu)^* \otimes V) \to \Sym((\Omega^\bu)^* \otimes V)
\]
that sends $\Phi \otimes x$ to $\iota(\Phi(x))$,
where 
\[
\iota: \Sym(\Omega_c^\bu \otimes V) \to \Sym((\Omega^\bu)^* \otimes V)
\]

4) Show this putative center fits into the commuting triangle and hence possesses a canonical map to the "true" center

5) Use the Francis-Lurie recognition principle to show this map is a quasi-iso
\end{proof}
