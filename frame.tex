\documentclass[11pt]{amsart}

\usepackage{macros}
\usepackage[bbgreekl]{mathbbol}

\linespread{1.25}

%\usepackage[final]{pdfpages}

\setcounter{tocdepth}{2}
\numberwithin{equation}{section}



\def\brian{\textcolor{blue}{BW: }\textcolor{blue}}
\def\owen{\textcolor{magenta}{OG: }\textcolor{magenta}}
\def\mahmoud{\textcolor{green}{MZ: }\textcolor{green}}


\begin{document}
\title{Centralizers of higher enveloping algebras and their large $N$ limits}

%\author{Owen Gwilliam}
%\address{Department of Mathematics and Statistics \\
%Lederle Graduate Research Tower, 1623D \\
%University of Massachusetts Amherst \\
%710 N. Pleasant Street}
%\email{gwilliam@math.umass.edu}
%
%\author{Brian Williams}
%\address{Department of Mathematics, 
%Northeastern University \\ 
%567 Lake Hall \\ 
%Boston, MA 02115 \\ U.S.A.}
%\email{br.williams@northeastern.edu}


\maketitle
\thispagestyle{empty}

\tableofcontents
 
Recently there has emerged a systematic generalization of the enveloping algebra $U\fg$ of a Lie algebra $\fg$ to {\em higher} enveloping algebras:
each dg Lie algebra $\fg$ has an enveloping $E_n$ algebra $\UU_n\fg$.
Building on prior work \cite{BD, AF}, Knudsen \cite{Knudsen} has developed these ideas extensively and offered useful models as factorization algebras.
Our central goal here is to document generalizations of two important results about enveloping algebras of Lie algebras: 
\begin{enumerate}
\item[(1)] the (underived) center of $U\sl_{N+1}$ is isomorphic to $(\Sym(\CC^{N}))^{S_N}$, where we view $\CC^{N}$ as the Cartan subalgebra of $\sl_{N+1}$ and $S_N$ as the Weyl group of $SL_{N+1}$, and
\item[(2)] the derived center of the enveloping algebra $U\fg$ is given by derived invariants of the adjoint action of $\fg$ on $U\fg$:
\[
\Hoch^*(U\fg,U\fg) \simeq \clies(\fg, U\fg),
\]
\end{enumerate}
but in a way that works uniformly for these higher enveloping algebras.

Our results include a cousin of these results where we vary (1) by taking the limit as $N \to \infty$ and vary (2) by studying the derived {\em centralizer} of a map $\UU_n\sl_\infty \hookrightarrow \UU_n\gl_\infty(A)$, where $A$ is a dg algebra.
(The derived center is the centralizer of the identity map $\UU_n\fg \to \UU_n\fg$.)
These notions will be reviewed and explained in the next section below,
where we also describe our results in detail.
When $n=1$, however, note that we find that \owen{state the classical thing here!}

Our main tool for this result is the Loday-Quillen-Tsygan (LQT) theorem,
which describes the large $N$ limit of Lie algebra homology of $\gl_N(A)$ for $A$ a dg algebra.
Experts will know that enveloping $E_n$ algebras can be constructed using Chevalley-Eilenberg chains
(i.e., that they are shifted versions of $\cliels$, in a precise sense),
and hence will recognize the relevance of the LQT theorem in this context.
From this perspective, our main result can be understood as an $E_n$ generalization of the LQT theorem.

We hope these constructions and ideas have interesting applications to other large $N$ phenomena,
such as in quantum field theory and random matrix theory.

\begin{rmk}
Throughout this document, we will work with differential graded vector spaces over a field $k$ of characteristic zero.
When we write ``Lie algebra,'' we mean a dg Lie algebra; if we mean an ordinary Lie algebra (such as $\sl_2(\CC)$), we say so explicitly.
Similarly, ``algebra'' means ``dg algebra'' unless stated otherwise.
Our theorems will be stated in terms of the stable $\infty$-category of such cochain complexes (one is free to work with the appropriate stable model or pretriangulated dg category, as one wishes).
Many of our constructions involve, however, explicit manipulations with standard, explicit complexes.
To minimize confusion, we try to indicate the level at which we are working but that may be left implicit in some places.
\end{rmk}

\section{Our main results}

See section~\ref{sec: enveloping} for a summary of the enveloping $E_n$ algebra construction, 
and see section~\ref{sec: centralizer} for a summary of centralizers of maps of $E_n$ algebras.

\begin{thm}
\label{thm: centralizer}
Let $f: \UU_n \fg \to B$ denote a map of $E_n$ algebras, where $\UU_n \fg$ is the enveloping $E_n$ algebra of a dg Lie algebra $\fg$.
In particular, $B$ is an $E_n$ algebra in $\fg$-modules via the map~$f$.

The centralizer ${\frak Z}_{E_n}(f)$ is the $E_n$ algebra $\clies(\fg, B)$.
\end{thm}

By using Knudsen's results, we have the following consequence.

\begin{cor}
Let $\rho: \fg \to \fh$ be a map of dg Lie algebras, and let $\UU_n \rho$ denote the associated map of enveloping $E_n$ algebras.
When $k$ is $\RR$ or $\CC$, 
the centralizer ${\frak Z}_{E_n}(\UU_n \rho)$ is modeled by the locally constant factorization algebra that assigns
\[
\clies(\fg \otimes \Omega^*(U), \cliels( \fh \otimes \Omega^*_c(U)))
\]
to each open set $U \subset \RR^n$.
\end{cor}

An immediate corollary, by taking the identity map, is the following.

\begin{cor}
\label{thm: center}
The center of the enveloping $E_n$ algebra $\UU_n\fg$ is the $E_n$ algebra $\clies(\fg, \UU_n\fg)$.
(It has, of course, a natural $E_{n+1}$ algebra structure.) 

When $k$ is $\RR$ or $\CC$, 
it is modeled by the locally constant factorization algebra that assigns
\[
\clies(\fg \otimes \Omega^*(U), \cliels( \fg \otimes \Omega^*_c(U)))
\]
to each open set $U \subset \RR^n$.
\end{cor}

This result generalizes the fact that $\Hoch^*(U\fg,U\fg) \simeq \clies(\fg, U\fg)$, since the associative algebra $U\fg$ is modeled by the locally constant one-dimensional factorization algebra that assigns $\cliels (\fg \otimes \Omega^*_c(U))$ to each open $U \subset \RR$. 

Having an explicit presentation for the centralizer of a map of $E_n$ algebras allows for immediate calculations of factorization homology.
Indeed, for $M$ a closed $n$-dimensional manifold,
\[
\int_M {\frak Z}_{E_n}(\UU_n \rho) \simeq \clies(\fg \otimes \Omega^*(M), \cliels( \fh \otimes \Omega^*(M))).
\]
When $M = S^n$, we thus see
\[
\int_{S^n} {\frak Z}_{E_n}(\UU_n \rho) \simeq \clies(\fg[\epsilon_n], \cliels( \fh[\epsilon_n]))
\]
as $\Omega^*(S^n) \simeq \CC[\epsilon_n]$ where $\epsilon_n$ is a formal variable with $|\epsilon_n| = n$ and $\epsilon_n^2 = 0$.
In the case that $n=1$, this recovers a theorem of Tamarkin-Tsygan \cite{TT} on the Hochschild cohomology of differential operators in the case of a group $G$ (see Example \ref{eg: TT}). 

\begin{rmk}
The preceding results are well-known to experts in factorization homology (we discussed it with several, including Francis and Knudsen),
since they are quick consequences of published work, as will be seen below.
But these statements seem not to have be documented yet in the literature, 
and so we record them here.
\end{rmk}

Our primary large $N$ result is the following, which is the novel contribution of this paper.

\begin{thm}
\label{thm: large N}
Let $A$ be a unital algebra over a field $k$ of characteristic zero. 
The map of Lie algebras
\[
u_A: \sl_\infty(k) \hookrightarrow \gl_\infty(A),
\]
induced by the unit in $A$, determines a map 
\[
\UU_nu_A: \UU_n\sl_\infty(k) \to \UU_n\gl_\infty(A).
\]
In particular, $\UU_n\gl_\infty(A)$ is an $E_n$ algebra in $\sl_\infty(k)$-representations.

The centralizer ${\frak Z}_{E_n}(\UU_nu_A)$ of this map is the $E_n$ algebra obtained by forgetting the dg commutative algebra
\[
\clies(\sl_\infty(k)) \otimes \Sym(\Cyc_*(A)[1-n])
\]
down to an $E_n$ algebra.

When $k$ is $\RR$ or $\CC$, 
it is modeled by the locally constant factorization algebra that assigns
\[
\clies(\sl_\infty(\Omega^*(U)))\otimes \Sym(\Cyc_*(A \otimes \Omega^*_c(U))[1])
\]
to each open set $U \subset \RR^n$.
\end{thm}

\owen{We should write explicitly what happens in the case $A = k$, and maybe $n=1$.}

The explicit factorization algebras provide convenient methods for computing factorization homology.
As shown in \owen{exact reference} in \cite{Knudsen}, 
the enveloping $E_n$ algebras are homotopy $O(n)$-fixed points and hence determine factorization algebras on unoriented $n$-dimensional manifolds.

\section{Enveloping $E_n$ algebras of Lie algebras}

Although \cite{Knudsen} develops an adjunction between Lie and $E_n$ algebras in any presentable stable symmetric monoidal $\infty$-category $\cC$,
we need his result only in the setting of chain complexes over a field $k$ of characteristic zero $\Mod_k$ with tensor product~$\otimes_k$.
We summarize his result in that setting as follows.

\def\oblv{{\rm oblv}}
\def\Free{{\rm Free}}

\begin{thm}
There is a commuting square of adjunctions
\[
\begin{tikzcd}
\Alg_\Lie(\Mod_k) \arrow[r, bend left=20, shift left=.5ex,"\UU_n"] \arrow[dd] & \Alg_{E_n}(\Mod_k) \arrow[l, shift left=.5ex, "\oblv"] \arrow[dd]\\
&\\
\Mod_k \arrow[r, bend left=20, shift left=.5ex, "\Sigma^{1-n}"] \arrow[uu, bend left=20, shift left=.5ex, "\Free_\Lie"] & \Mod_k \arrow[l, shift left=.5ex,"\Sigma^{n-1}"] \arrow[uu, bend right=20, shift right=.5ex, "\Free_{E_n}"']
\end{tikzcd}
\]
where $\Sigma$ denotes suspension \owen{because tikzcd does weird things with ``[1-n]'' instead}.
(Curving arrows denote left adjoints.)

When $k$ is $\RR$ or $\CC$, 
the enveloping $E_n$ algebra $\UU_n \fg$ of a dg Lie algebra $\fg$ is modeled by the locally constant factorization algebra that assigns
\[
\cliels(\Omega^*_c(U) \otimes \fg)
\]
to each open set $U \subset \RR^n$.
\end{thm}

The second part of this result implies the explicit models in our theorems.

\section{Centralizers of maps of $E_n$ algebras and the derived center}
\label{sec: centralizer}

We provide a gloss of the main notions here but refer the reader to Section 5.3 of \cite{LurieHA} and \cite{FrancisHH} for extensive and detailed discussions.

Let $\cC^\otimes$ denote a presentable stable symmetric monoidal $\infty$-category (for us, chain complexes over a field $k$ of characteristic zero with~$\otimes_k$).

\begin{dfn}
For a map $f: A \to B$ of $E_n$ algebras in $\cC$, 
the {\em centralizer} ${\frak Z}_{E_n}(f)$ is the $E_n$ algebra in $\cC$ that is universal among those that sit in a commuting diagram
\[
\begin{tikzcd}
& {\frak Z}_{E_n}(f) \otimes A \arrow[dr] & \\
A \arrow[ur, "u"] \arrow[rr, "f"] && B
\end{tikzcd}
\]
where $u$ is induced by the identity map $1_\cC \to {\frak Z}_{E_n}(f)$.

The {\em center} of an $E_n$ algebra $A$ is the centralizer of its identity morphism $\id_A: A \to A$.
\end{dfn}

When $n=1$ and $\cC$ just ordinary vector spaces, the centralizer of $f: A \to B$ thus consists of the subspace of elements of $B$ that commute with the image $f(A)$ inside~$B$.
The center of $A$ thus recovers the traditional notion.

An alternative characterization is found in Theorem 5.3.1.30 of~\cite{LurieHA}.

\begin{prp}
The centralizer ${\frak Z}_{E_n}(f)$ can be identified with the classifying object for maps from $A$ to $B$ in the $\infty$-category $\Mod^{E_n}_A(\cC)$.
\end{prp}

When $n=1$ and $\cC$ is chain complexes over the field $k$, 
then $\Mod^{E_1}_A(\cC)$ is equivalent to $A-A$-bimodules and hence ${\frak Z}_{E_1}(f) \simeq \HH^*(A,B_f)$,
where $B_f$ denotes $B$ viewed as an $A-A$-bimodule via the map~$f$.

We record a key fact drawn from \cite{LurieHA} and~\cite{FrancisHH}, which generalizes this situation.

\begin{prp}
Let $\cC^\otimes$ be a presentable symmetric monoidal stable $\infty$-category whose tensor product preserves colimits in each variable separately.
For $A$ be an $E_n$ algebra in $\cC$,
the $\infty$-category $\Mod^{E_n}_A(\cC)$ is equivalent to the $\infty$-category $\LMod_{\int_{S^{n-1} A}}(\cC)$.
\end{prp}

See Remark 7.3.5.3 of \cite{LurieHA} and Section 2 of \cite{FrancisHH}.
Concretely, this claim arises from the fact that as a module over $A$ as an $E_n$ algebra, 
$M \in \Mod^{E_n}_A(\cC)$ is equipped with a commuting family of actions by $A$ parametrized by $\RR^n \setminus \{0\}$.
This action factors through the left action of the $E_1$ algebra $\int_{S^{n-1} \times \RR} A$ arising from factorization homology over the complement of the origin,
which integrates all those actions and through which they all factor.
As shown in Theorem 7.5.3.1 of \cite{LurieHA} and earlier in Proposition 3.16 of \cite{FrancisHH}, 
this algebra $\int_{S^{n-1} \times \RR} A$ is equivalent to ${\rm Free}(1_\cC)$, 
the image of the unit object $1_\cC \in \cC$ under the left adjoint ${\rm Free}$ to the forgetful functor $\Mod^{E_n}_A(\cC) \to \cC$.

We now use these results to prove the central claim of Theorem~\ref{thm: centralizer}.

\section{Proof of Theorem \ref{thm: centralizer}}

Let $\fg$ be a dg Lie algebra, let $B$ be an $E_n$ algebra, and let $f: \UU_n \fg \to B$ be a map of $E_n$ algebras.
Then the results of the preceding section imply
\begin{align*}
{\frak Z}_{E_n}(f) &\simeq {{\rm Mor}}_{\Mod^{E_n}_{\UU_n}}(\UU_n \fg, B_f) \\
&\simeq {\rm Mor}_{\LMod_{\int_{S^{n-1}} \UU_n}}(\UU_n \fg, B_f)
\end{align*}
where 
\begin{itemize}
\item $B_f$ denotes the object in $\Mod^{E_n}_{\UU_n}$ or, equivalently, in $\LMod_{\int_{S^{n-1}}\UU_n}$ determined by~$f$,
\item ${\rm Mor}_{\Mod^{E_n}_{\UU_n}}(\UU_n \fg, B)$ denotes the classifying object for maps from $\UU_n$ to $B_f$ in the $\infty$-category $\Mod^{E_n}_{\UU_n}$ (i.e., an internal mapping object),
\item and ${\rm Mor}_{\LMod_{\int_{S^{n-1}} \UU_n}} (\UU_n \fg, B_f)$ denotes the classifying object for maps from ${\UU_n}$ to $B$ in the $\infty$-category $\LMod_{\int_{S^{n-1}}\UU_n}$.
\end{itemize}
Observe now that $\int_{S^{n-1}}\UU_n$ is an $E_1$ algebra equivalent to the dg algebra $U(\fg \otimes k[\epsilon_{n-1}])$.
This is because the rational $(n-1)$-sphere is formal, and hence its cochains are quasi-isomorphic to its cohomology $k[\epsilon_{n-1}]$.
Using the canonical inclusion $i: U\fg \to U(\fg \otimes k[\epsilon_{n-1}])$,
we find
\[
k \otimes_{U\fg} U(\fg \otimes k[\epsilon_{n-1}]) \simeq \Sym(\fg[1-n]).
\]
Recall now that $\UU_n \fg \simeq \Sym(\fg[1-n])$ as $\fg$-modules.
Base change thus implies
\[
{\rm Mor}_{\LMod_{\int_{S^{n-1}} \UU_n}} (\UU_n \fg, B_f)  \simeq {\rm Mor}_{U\fg} (k, B_f).
\]
The right hand side is precisely Lie algebra cochains of $\fg$ with coefficients in~$B_f$, as desired.

\brian{do you like this example?}
\begin{eg}\label{eg: TT}
Consider the case $n=1$ and $B = U \fg$ where $f$ is simply the identity map. 
The Hochschild cohomology of $U \fg$ can be expressed in terms of Lie algebra cohomology as $\Hoch^*(U\fg,U\fg) \simeq \clies(\fg, U\fg)$. 
On the other hand, our model for the derived center ${\frak Z}(U(\fg))$ of $U \fg$ is the one-dimensional factorization algebra assigning 
\[
\clies(\fg \otimes \Omega^*(U), \cliels( \fg \otimes \Omega^*_c(U)))
\]
to each open set $U \subset \RR$.
In particular, the factorization homology of the derived center along the closed manifold $S^1$ can be computed as
\[
\int_{S^1} {\frak Z}(U(\fg)) \simeq \clies(\fg[\epsilon], \cliels( \fg[\epsilon])),
\]
where $\epsilon$ is a formal variable of degree $+1$. 
\brian{let's talk about this.}
\end{eg}

\section{A variant of the Loday-Quillen-Tsygan theorem}

Throughout this section, let $A$ be a dg algebra over a field $k$ of characteristic zero.
We do not assume $A$ is unital; whenever unitality leads to a stronger result, we will explicitly add it as a hypothesis.
(It is possible to work with $A_\infty$ algebras rather than dg algebras, 
but it changes nothing structurally essential while adding a distracting layer of notational complexity.)

\begin{dfn}
Let $\cC(A)$ denote the cochain complex $\Cyc_*(A)[1]$, which may be viewed as an abelian Lie algebra. 
Let $\cL(A)$ denote the dg Lie algebra $\gl_\infty(A)$.
\end{dfn}

Let $\sl_\infty$ denote $\sl_\infty(k)$, the trace-zero matrices of rank infinity with only finitely many nonzero entries in~$k$.
Let $\gl_\infty$ denote $\gl_\infty(k)$, the matrices of rank infinity with only finitely many nonzero entries in~$k$.
Let $\gl_\infty(A)$ denote the matrices of rank infinity with only finitely many nonzero entries in~$A$.
There is a natural action of $\sl_\infty$ on $\gl_\infty(A)$, 
and hence on its Chevalley-Eilenberg chains $\cliels(\cL(A))$.
A key step in the LQT theorem is the construction of a quasi-isomorphism
\beqn\label{lqtco}
\lambda^{co}_A: \cliels(\cL(A))_{\sl_\infty} \xto{\cong} \Sym(\cC(A)).
\eeqn
When $A$ is unital, the inclusion of the identity $k \hookrightarrow A$ means the action of $\sl_\infty$ on $\gl_\infty(A)$ is inner, 
and hence the projection $\cliels(\cL(A)) \to \cliels(\cL(A))_{\sl_\infty}$ is a quasi-isomorphism.
The standard statement of the LQT theorem composes these two maps.

\begin{thm}
For $A$ a unital dg algebra over a characteristic zero field $k$,
there is a quasi-isomorphism
\beqn\label{lqt}
\lambda_A: \cliels(\cL(A)) \xto{\simeq} \Sym(\cC(A))
\eeqn
of dg cocommutative Hopf algebras.
\end{thm}

For us it is the coalgebra structure that is particularly important, although the full Hopf algebra structure is typically emphasized.

As the constructions are natural in the algebra $A$, one can phrase a stronger but more abstract statement.

\begin{cor}
There is a natural transformation
\[
\lambda: \cliels(\cL(-)) \Longrightarrow \Sym(\cC(-))
\]
between these two functors from the dg category of unital dg algebras to the dg category of dg cocommutative Hopf algebras over~$k$.
It is a quasi-isomorphism on every algebra.
\end{cor}

For our purposes, a variant is more useful, but this variant follows by easy, formal arguments.

\def\lqt{{\bblambda}}

\begin{thm}
\label{thm: variant LQT}
There is a natural transformation
\[
\lqt: \clies(\sl_\infty, \cliels(\cL(-)))  \Longrightarrow \clies(\sl_\infty) \otimes \Sym(\cC(-))
\]
between these two functors from the dg category of {\em non}unital dg algebras to the dg category of dg Hopf algebras over~$k$.
For each nonunital dg algebra $A$, the map
\beqn\label{lqtinv}
\lqt_A: \clies(\sl_\infty, \cliels(\cL(A))) \xto{\simeq} \clies(\sl_\infty) \otimes \Sym(\cC(A))
\eeqn
is a quasi-isomorphism.
\end{thm}

This result yields the central claim of Theorem~\ref{thm: large N}, about the large $N$ limit of the centralizer.

\begin{rmk}
Loday and Procesi proved versions of the Loday-Quillen-Tsygan theorem for the Lie algebras $\mathfrak{o}_n$ and $\mathfrak{sp}_{2n}$,
in which cyclic homology of the algebra is replaced by its dihedral homology.
As nothing substantive changes in proving analogous versions of our results above, 
we do not spell out the details here.
\end{rmk}

\begin{proof}
We take for granted that the reader is familiar with the proof of the LQT theorem as articulated in \cite{LQ, Weibel, LodayCyclic}.
Our primary modification is to replace the coinvariants $\cliels(\cL(A))_{\sl_\infty}$ by the invariants $\cliels(\cL(A))^{\sl_\infty}$.
Note that there is a canonical map
\beqn\label{invtoco}
\cliels(\cL(A))^{\sl_\infty} \hookrightarrow \cliels(\cL(A)) \to \cliels(\cL(A))_{\sl_\infty},
\eeqn
and in fact this map is an isomorphism, which we now show.

To see this claim, we need to understand how $\sl_\infty$ acts on the underlying graded module of $\cliels(\cL(A))$. Note that as a $k$-module, $\cL(A) \cong \gl_\infty \otimes A$, which is a free $k$-module. 
Each $n$th tensor power thus factors as
\[
(\cL(A)[1])^{\otimes n} \cong (\gl_\infty[1])^{\otimes n} \otimes A^{\otimes n}.
\]
Taking the coinvariants under the action of the symmetric group $S_n$
(which acts diagonally on the $\gl_\infty$ factor and the $A$ factor),
we obtain
\[
\Sym^n (\gl_\infty[1]) \otimes A.
\]
But the action of $\sl_\infty$ is then wholly on the left hand factor, leaving $A$ independent.

There is a natural decomposition of $\Sym^n (\gl_\infty[1])$ into a direct sum $T_n \oplus T_n^\perp$ of representations of $\sl_\infty$, where $T_n$ is trivial and the complement $T_n^\perp$ has no trivial components.
If we replace $\gl_\infty$ with the finite-dimensional Lie algebra $\gl_m$, 
the analogous decomposition of $\Sym^n (\gl_m[1])$ exists by semisimplicity. 
In fact, with $m > n$, invariant theory implies
\[
(\gl_m^{\otimes n})^{\sl_m} = (\gl_m^{\otimes n})_{\sl_m} \cong k S_n,
\]  
the group algebra of the symmetric group $S_n$.
A stabilization phenomenon then implies the decomposition as $m \to \infty$.
We note that we see here the equivalence of the $\sl_m$-invariants with the $\sl_m$-coinvariants,
which holds for every degree $n$ and hence yields the desired isomorphism~(\ref{invtoco}).
\owen{MZ points out we should show the filtered colimit here commutes with taking invariants (which is a limit), 
which should be true since filtered colimits commute with finite limits.}

We thus conclude that there is a decomposition
\[
\cliels(\cL(A)) = T(A) \oplus T^\perp(A)
\]
into a direct sum of cochain complexes,
where the first summand is a trivial $\sl_\infty$-module and the second has no trivial components.
We will not unravel explicitly how to construct $T(A)$ from the algebra $A$ and the modules~$T_n$,
but the reader should recognize that $T(A) = \cliels(\cL(A))^{\sl_\infty}$.

We now recall two familiar facts about the Lie algebra cohomology of a semisimple Lie algebra~$\fg$:
\begin{itemize}
\item the Lie algebra cohomology of an irreducible, nontrivial, finite-dimensional module $M$ is zero, and
\item for any trivial module $N$, there is a natural isomorphism $H^*_\Lie( \fg,N) \cong H^*_\Lie(\fg) \otimes N$.
\end{itemize}
Together these imply that for an arbitrary finite-dimensional module~$M$,
\[
\clies(\fg,M) \simeq \clies(\fg) \otimes M^\fg.
\]
Thus, we have a decomposition
\[
\clies(\sl_\infty,\cliels(\cL(A))) = \clies(\sl_\infty,T(A)) \oplus \clies(\sl_\infty,T^\perp(A)), 
\]
and there is a split surjection
\[
\bbsigma_A: \clies(\sl_\infty,\cliels(\cL(A))) \to \clies(\sl_\infty,T(A)) \cong \clies(\sl_\infty) \otimes T(A) 
\]
that is a quasi-isomorphism.

We define our desired map $\lqt$ as the composite
\[
\lqt_A = \left(\id_{\clies(\sl_\infty)} \otimes \lambda^{co}_A \right) \circ \bbsigma_A,
\]
which is a quasi-isomorphism by construction.

Our arguments, like those for the LQT theorem, are natural in the algebra~$A$ and hence we have constructed the natural transformation $\lqt$ implicitly.
\end{proof}

\section{A factorization algebra arising from cyclic homology}

We explain here how to construct a factorization algebra using cyclic homology,
much as Lie algebra homology leads to an explicit factorization Lodayalgebra modeling the enveloping $E_n$ algebra. 
Throughout this section, let $k$ denote the field $\RR$ or $\CC$; 
by mimicking \cite{Knudsen}, one could also produce an explicit model over~$\QQ$.

\begin{dfn}
Let $A$ be a dg algebra over $k$, and let $\cA = A \otimes \Omega^*_c$ denote the associated cosheaf of dg algebras on~$\RR^n$.
Let $\cC_A$ denote the precosheaf $\Cyc_*(\cA)[1]$ of cochain complexes on $\RR^n$,
which assigns to each open set $U$, the 1-shifted cyclic chains on $\cA(U)$.
\end{dfn}

Note that for two algebras $B$ and $B'$,
\[
\Cyc_*(B) \oplus \Cyc_*(B') \simeq \Cyc_*(B \times B') .
\] 
The corresponding statement for ordinary Hochschild homology is induced by the projections $B \leftarrow B \times B' \to B'$, and any map of algebras that induces a quasi-isomorphism on Hochschild homology induces one on cyclic homology as well.
See, for instance, 2.2.12 of \cite{LodayCyclic}. 
Hence, for any finite collection of disjoint opens $U_1$, \dots, $U_k$, 
we have
\[
\cC_A(U_1 \sqcup \cdots \sqcup U_k) \simeq \cC_A(U_1) \oplus \cdots \oplus \cC_A(U_k).
\]
(In this sense $\cC_A$ behaves like a cosheaf, up to quasi-isomorphism. 
\owen{Does cyclic homology preserve filtered colimits? If so, I think $\cC_A$ may be a homotopy cosheaf.}
\brian{Yes, see E.2.1.1 of \cite{LodayCyclic} on p. 59.}
This property underlies the following claim.


\begin{lmm}
\label{lem: cycfact}
The functor $\Sym(\cC_A)$ sending an open set $U \subset \RR^n$ to $\Sym(\cC_A(U))$ is a locally constant prefactorization algebra.
\end{lmm}

%\begin{rmk}
%We note that, as with the definition of the Chevalley-Eilenberg chains of a local Lie algebra,
%we use here a construction of cyclic chains that plays nicely with the kind of vector spaces relevant to this situation,
%namely smooth sections of vector bundles.
%Where the cyclic quotient $A^{\otimes n}/C_n$ would appear for an ordinary algebra in complex vector spaces,
%we take the $\Omega^{0,*}(X^n)/C_n$ and so on.
%\owen{I need to check that the $\Sym$ doesn't lead to issues \dots If we must, we can ignore the quasi-isomorphism and focus on the map just to cyclic homology.}
%\end{rmk}

\begin{proof}
It remains to provide the structure maps of the putative prefactorization algebra.
We need to provide a multilinear structure map 
\beqn
\label{eqn: desiredmap}
\cC(U_1) \times \cdots \times \cC(U_n) \to \cC(V)
\eeqn
for any finite set of pairwise disjoint opens $U_1,\ldots, U_n$ inside a larger open $V$.
For the cosheaf $\Omega^{0,*}_c$ on pairwise disjoint opens $U_1,\ldots, U_n$,
the isomorphism of dg algebras
\[
\Omega^{*}_c(U_1) \oplus \cdots \oplus \Omega^{*}_c(U_n) \cong \Omega^{*}_c(U_1 \sqcup \cdots \sqcup U_n),
\]
determines a quasi-isomorphism
\beqn
\label{eqn:cyccosheaf}
\Cyc_*(A \otimes \Omega^{*}_c(U_1)) \oplus \cdots \oplus \Cyc_*(A \otimes \Omega^{*}_c(U_n)) \xto{\simeq} \Cyc_*(A \otimes \Omega^{*}_c(U_1 \sqcup \cdots \sqcup U_n)).
\eeqn
The inclusion $U_1 \sqcup \cdots \sqcup U_n \hookrightarrow V$ then provides a map
\[
\Cyc_*(A \otimes \Omega^{*}_c(U_1 \sqcup \cdots \sqcup U_n)) \to \Cyc_*(A \otimes \Omega^{*}(V)),
\]
and so determines a map
\beqn
\label{eqn:map2}
\Sym(\cC_A(U_1 \sqcup \cdots \sqcup U_n)) \to \Sym(\cC_A(V)).
\eeqn
Likewise, applying $\Sym$ to map~\eqref{eqn:cyccosheaf} provides
\[
\Sym(\cC_A(U_1)) \otimes \cdots \otimes \Sym(\cC_A(U_n)) \to \Sym(\cC_A(U_1 \sqcup \cdots \sqcup U_n)).
\]
We thus obtain the desired map \eqref{eqn: desiredmap} as a composite.
This construction is automatically associative for nested inclusions of pairwise disjoint opens,
and so $\Sym(\cC_A)$ is a prefactorization algebra.
\end{proof}

\begin{cor}
For $A$ a dg algebra over $k$, Theorem~\ref{thm: variant LQT} provides a weak equivalence of prefactorization algebras
\[
\lqt: \clies(\sl_\infty(\Omega^*), \cliels(\cL(\cA))) \xto{\simeq} \clies(\sl_\infty(\Omega^*)) \otimes \Sym(\cC_A).
\]
Hence, the functor $\Sym(\cC_A)$ determines a locally constant factorization algebra on $\RR^n$, and hence an $E_n$ algebra.
\end{cor}

This result yields the latter part of Theorem~\ref{thm: large N}, about an explicit factorization algebra.

\begin{rmk}
As remarked earlier, the Loday-Procesi theorem for the Lie algebras $\mathfrak{o}_n$ and $\mathfrak{sp}_{2n}$,
in which cyclic homology of the algebra is replaced by its dihedral homology,
lead to an analogous class of factorization algebras.
\end{rmk}

%\section{Proof of Theorem~\ref{thm: large N}}


\bibliographystyle{alpha}  
\bibliography{centralize}

%%\bibliographystyle{spmpsci}  

\end{document}