


\documentclass[11pt]{amsart}

\usepackage{macros}
\usepackage[bbgreekl]{mathbbol}

\linespread{1.25}

%\usepackage[final]{pdfpages}

\setcounter{tocdepth}{2}
\numberwithin{equation}{section}



\def\brian{\textcolor{blue}{BW: }\textcolor{blue}}
\def\owen{\textcolor{magenta}{OG: }\textcolor{magenta}}
\def\mahmoud{\textcolor{olive}{MZ: }\textcolor{olive}}
\def\greg{\textcolor{red}{GG: }\textcolor{red}}


\begin{document}
\title{Centralizers of higher enveloping algebras and their large $N$ limits}

%\author{Owen Gwilliam}
%\address{Department of Mathematics and Statistics \\
%Lederle Graduate Research Tower, 1623D \\
%University of Massachusetts Amherst \\
%710 N. Pleasant Street}
%\email{gwilliam@math.umass.edu}
%
%\author{Brian Williams}
%\address{Department of Mathematics, 
%Northeastern University \\ 
%567 Lake Hall \\ 
%Boston, MA 02115 \\ U.S.A.}
%\email{br.williams@northeastern.edu}


\maketitle
\thispagestyle{empty}

\tableofcontents
 
Recently, a generalization of the concept of the enveloping algebra of a Lie algebra has emerged: each dg Lie algebra $\fg$ has an enveloping $E_n$ algebra $\UU_n\fg$, for every $n$, so that $\UU_1\fg$ is the usual enveloping algebra $U\fg$. Building on prior work \cite{BD, AF} [Owen's thesis?], Knudsen \cite{Knudsen} developed these ideas extensively and offered a useful factorization algebra model for the enveloping $E_n$ algebra. Our central goal here is to generalize the following results about the enveloping algebra $U\fg$ to $\UU_n\fg$, for all $n$.

\begin{enumerate}
\item[(1)] the (underived) center of $U\sl_{N+1}$ is isomorphic to $(\Sym(\CC^{N}))^{S_N}$, where $\CC^{N}$ denotes the Cartan subalgebra of $\sl_{N+1}$ and $S_N$ is the Weyl group of $SL_{N+1}$, and
\item[(2)] the derived center of the enveloping algebra $U\fg$ is given by derived invariants of the adjoint action of $\fg$ on $U\fg$. That is to say $\Hoch^*(U\fg,U\fg) \simeq \clies(\fg, U\fg)$.
\end{enumerate}

Our results pertains to (1) by taking $N \to \infty$, and to (2) through the derived {\em centralizer} of the map $\UU_n\sl_\infty \hookrightarrow \UU_n\gl_\infty(A)$, where $A$ is a dg algebra. Note that the usual derived center is the centralizer of the identity map $\UU_n\fg \to \UU_n\fg$.

We will review these notions and describe our results in detail in the next section. When $n=1$,  \owen{state the classical thing here!}

Our primary tool here is the Loday-Quillen-Tsygan (LQT) theorem which describes the large $N$ limit of the Lie algebra homology of $\gl_N(A)$, for a dg associative algebra $A$. Since the enveloping $E_n$ algebra of a Lie algebra can be constructed as a locally constant factorization algebra made out of the shifted Chevalley-Eilenberg chains \cite{}, the relevance of the LQT theorem in this context  should be apparent. From this point of view, one can think of our main result as an $E_n$ generalization of the LQT theorem. We hope these ideas have exciting applications in quantum field theory and random matrix theory, as well as to other large $N$ limit phenomena.

\begin{rmk}
Throughout this document, we will work with differential graded vector spaces over a field $k$ of characteristic zero. We refer to dg Lie algebras simply as Lie algebra, and refer to those without grading and a differential, such as $\sl_2(\CC)$, as ordinary Lie algebras. Similarly, the term algebra means dg algebra unless stated otherwise.
\end{rmk}

\begin{rmk}
We state our theorems in terms of the stable $\infty$-category cochain complexes instead of the appropriate stable model categories or pre-triangulated dg categories. Many of our constructions, however, involve explicit manipulations of standard complexes, and when it is helpful, we indicate the level at which we are working.
\end{rmk}





\section{Our main results}

See section~\ref{sec: enveloping} for a summary of the enveloping $E_n$ algebra construction, and see section~\ref{sec: centralizer} for a summary of centralizers of maps of $E_n$ algebras.

\begin{thm}
\label{thm: centralizer}
Let $\UU_n \fg$ be the enveloping $E_n$ algebra of a dg Lie algebra $\fg$ and $f: \UU_n \fg \to B$ a map of $E_n$ algebras which gives $B$ the structure of an $E_n$ algebra in $\fg$-modules. Then, the centralizer ${\frak Z}_{E_n}(f)$ is the $E_n$ algebra $\clies(\fg, B)$.
\end{thm}

\greg{shall explain/remind somewhere what is the $E_n$-structure there: namely induced by $E_n$-structure of $B$ and commutative multiplication in $\clies$, namely convolution between an $E_\infty$-coalgebra and an $E_n$-algebra}

By Knudsen's results \cite{Knudsen}, we have the following consequence.

\begin{cor}
Let $k$ is $\RR$ or $\CC$, and assume $\UU_n \rho:  \UU_n \fg \to \UU_n \fh$ is induced by a map of Lie algebras $\rho: \fg \to \fh$.  Then, the centralizer ${\frak Z}_{E_n}(\UU_n \rho)$ is modeled by the locally constant factorization algebra that assigns
\[
\clies(\fg \otimes \Omega^*(U), \cliels( \fh \otimes \Omega^*_c(U)))
\]
to each open set $U \subset \RR^n$. Here, $\Omega^*_c(U)$ denotes the commutative differential graded algebra (cdga) of compactly supported forms on $U$.
\end{cor}

If $\rho: \fg \to \fg$ is the identity map, we get the following immediate corollary.

\begin{cor}
\label{thm: center}
The center ${\frak Z}_{E_n}(\UU_n\fg)$ of the enveloping $E_n$ algebra $\UU_n\fg$ is the $E_n$ algebra $\clies(\fg, \UU_n\fg)$. 
When $k$ is $\RR$ or $\CC$, the 
center is modeled by the locally constant factorization algebra that assigns
\[
\clies(\fg \otimes \Omega^*(U), \cliels( \fg \otimes \Omega^*_c(U)))
\]
to each open set $U \subset \RR^n$.
\end{cor}

In fact, ${\frak Z}_{E_n}(\UU_n\fg)$ has a natural $E_{n+1}$ algebra structure, 
as shown in \cite{LurieHA, GTZ}, 
but we will not examine it here.

\greg{shall relate this to $P_n$-center of $\fg$. that shall also (but I have to think about the proof more), at least up to formality gives the $P_{n+1}$-structure on it. }
This result generalizes the fact that $\Hoch^*(U\fg,U\fg) \simeq \clies(\fg, U\fg)$, since the associative algebra $U\fg$ is modeled by the locally constant one-dimensional factorization algebra that assigns $\cliels (\fg \otimes \Omega^*_c(U))$ to each open $U \subset \RR$. 

As another consequence of the theorem, we note that we recover an explicit model for the Koszul dual to the enveloping algebra. 
Namely, take $B$ to be the unit $E_n$ algebra $k$ (alternatively, take the enveloping $E_n$ algebra of the trivial Lie algebra $\fh = 0$).

\begin{cor}
\label{thm: koszul}
The $E_n$ Koszul dual of the enveloping $E_n$ algebra $\UU_n\fg$ is the underlying $E_n$ algebra of the $\clies(\fg)$. 
When $k$ is $\RR$ or $\CC$, the 
center is modeled by the locally constant factorization algebra that assigns
\[
\clies(\fg \otimes \Omega^*(U)))
\]
to each open set $U \subset \RR^n$.
\end{cor}

Having an explicit presentation for the centralizer of a map of $E_n$ algebras leads to a calculation of factorization homology. The enveloping $E_n$ algebra is the homotopy $O(n)$-fixed points, or a $\mathrm{Disk}_n$-algebra in the sense of~\cite{AF}, and hence determines a factorization algebra on any un-oriented $n$-dimensional manifold \cite{Knudsen}. Therefore, for a closed $n$-dimensional manifold $M$, one has
\[
\int_M {\frak Z}_{E_n}(\UU_n \rho) \simeq \clies(\fg \otimes \Omega^*(M), \cliels( \fh \otimes \Omega^*(M))).
\]
Since $\Omega^*(S^n) \simeq \CC[\epsilon_n]$, where $\epsilon_n$ is a formal variable with $|\epsilon_n| = n$ and $\epsilon_n^2 = 0$, we have
\[
\int_{S^n} {\frak Z}_{E_n}(\UU_n \rho) \simeq \clies(\fg[\epsilon_n], \cliels( \fh[\epsilon_n]))
\]

For $n=1$, this recovers a theorem of Tamarkin and Tsygan \cite{TT} on the Hochschild cohomology of differential operators on a Lie group $G$; see Example \ref{eg: TT}. 

\begin{rmk}
The preceding results are quick consequences of published work and are well-known to experts we spoke to, including John Francis and Ben Knudsen. Nevertheless, we record these results so they are documented in the literature.
\end{rmk}

Our novel contribution in this paper is the following large $N$ result. Let $A$ be a unital algebra over a field $k$ of characteristic zero, and $u_A: \sl_\infty(k) \hookrightarrow \gl_\infty(A)$ be the map of Lie algebras determined by the unit in $A$. The induced map $\UU_nu_A: \UU_n\sl_\infty(k) \to \UU_n\gl_\infty(A)$ makes $\UU_n\gl_\infty(A)$ is an $E_n$ algebra in $\sl_\infty(k)$-modules.

\begin{thm}
\label{thm: large N}

The centralizer ${\frak Z}_{E_n}(\UU_nu_A)$ of the map $\UU_nu_A$ is the $E_n$ algebra obtained by forgetting the dg commutative algebra
\[
\clies(\sl_\infty(k)) \otimes \Sym(\Cyc_*(A)[1-n])
\]
down to an $E_n$ algebra.

When $k$ is $\RR$ or $\CC$, this centralizer is modeled by the locally constant factorization algebra that assigns
\[
\clies(\sl_\infty(\Omega^*(U)))\otimes \Sym(\Cyc_*(A \otimes \Omega^*_c(U))[1])
\]
to each open set $U \subset \RR^n$.
\end{thm}

\owen{We should write explicitly what happens in the case $A = k$, and maybe $n=1$.}

We can calculate the factorization homology of this factorization algebra explicitly.

\greg{need to add a notation/convention somewhere including that $\Sigma^n$ will be shift functor, convention for grading and so on}
\section{Enveloping $E_n$ algebras of Lie algebras}

Although \cite{Knudsen} develops an adjunction between Lie and $E_n$ algebras \mahmoud{non-unital?} in any presentable symmetric monoidal stable $\infty$-category $\cC$,
we only work in the infinity category of chain complexes $(\Mod_k, \otimes_k)$, where $k$ is a field of characteristic zero. We summarize Knudsen's result in this setting as follows.




\def\oblv{{\rm oblv}}
\def\Free{{\rm Free}}

\begin{thm}
There is a commuting square of adjunctions
\[
\begin{tikzcd}
\Alg_\Lie(\Mod_k) \arrow[r, bend left=20, shift left=.5ex,"\UU_n"] \arrow[dd, "\oblv"] & \Alg_{E_n}(\Mod_k) \arrow[l, shift left=.5ex, "\oblv"] \arrow[dd, swap, "\oblv"]\\
&\\
\Mod_k \arrow[r, bend left=20, shift left=.5ex, "\text{$[1-n]$}"] \arrow[uu, bend left=20, shift left=.5ex, "\Free_\Lie"] & \Mod_k \arrow[l, shift left=.5ex, "\text{$[n-1]$}"] \arrow[uu, bend right=20, shift right=.5ex, "\Free_{E_n}"']
\end{tikzcd}
\]
where $\oblv$ denotes a forgetful functor.
(Curving arrows denote left adjoints.)

When $k$ is $\RR$ or $\CC$, 
the enveloping $E_n$ algebra $\UU_n \fg$ of a dg Lie algebra $\fg$ is modeled by the locally constant factorization algebra that assigns
\[
\cliels(\Omega^*_c(U) \otimes \fg)
\]
to each open set $U \subset \RR^n$.
\end{thm}

The second part of this result leads to the explicit models in our theorems.

\section{Centralizers of maps of $E_n$ algebras and the derived center}
\label{sec: centralizer}

We briefly review some of the main notions and refer the reader to \cite{LurieHA} Section 5.3 and \cite{FrancisHH} for detailed discussions. Let $\cC^\otimes$ denote a presentable stable symmetric monoidal $\infty$-category (for us, chain complexes over a field $k$ of characteristic zero with~$\otimes_k$).

\begin{dfn}
For a map $f: A \to B$ of $E_n$ algebras in $\cC$, 
the {\em centralizer} ${\frak Z}_{E_n}(f)$ is the $E_n$ algebra in $\cC$ that is universal among those that fit in a commuting diagram
\[
\begin{tikzcd}
& {\frak Z}_{E_n}(f) \otimes A \arrow[dr] & \\
A \arrow[ur, "u"] \arrow[rr, "f"] && B
\end{tikzcd}
\]
where $u$ is induced by the identity map $1_\cC \to {\frak Z}_{E_n}(f)$. The {\em center} of an $E_n$ algebra $A$ is the centralizer of its identity morphism $\id_A: A \to A$.
\end{dfn}
When $n=1$ and $\cC$ is the category of vector spaces, the centralizer of $f: A \to B$ consists of the subspace of elements of $B$ that commute with the image $f(A)$ inside~$B$. Thus, the center of $A$ agrees with the traditional notion. See Theorem 5.3.1.30 of~\cite{LurieHA}, or Proposition 6.22 in~\cite{GTZ3}, for an alternative characterization,
as follows.

\begin{prp}
The centralizer ${\frak Z}_{E_n}(f)$ can be identified with the classifying object for maps from $A$ to $B$ in the $\infty$-category~$\Mod^{E_n}_A(\cC)$.
\end{prp} 

Here ``classifying object'' means an internal hom object, so that a map from $M$ to ${\frak Z}_{E_n}(f)$ in $\Mod^{E_n}_A(\cC)$ corresponds to a map from $M \otimes A$ to $B$ in $\Mod^{E_n}_A(\cC)$. 
(More carefully, there is an equivalence of the appropriate mapping spaces in that category.)
\owen{We should double check this.}

When $n=1$ and $\cC$ is chain complexes over the field $k$, then $\Mod^{E_1}_A(\cC)$ is equivalent to $A\otimes A^{op}$-modules and hence ${\frak Z}_{E_1}(f) \simeq {\rm Hoch}^*(A,B_f)$,
where $B_f$ denotes $B$ viewed as an $A\otimes A^{op}$-modules via the map~$f$.

We record a key fact drawn from \cite{LurieHA},~\cite{FrancisHH}  and \cite{GG-Notes}, which generalizes this situation.

\begin{prp}
Let $\cC^\otimes$ be a presentable symmetric monoidal stable $\infty$-category whose tensor product preserves colimits in each variable separately.
For $A$ be an $E_n$ algebra in $\cC$,
the $\infty$-category $\Mod^{E_n}_A(\cC)$ is equivalent to the $\infty$-category $\LMod_{\int_{S^{n-1} A}}(\cC)$.
\end{prp}

See Remark 7.3.5.3 of \cite{LurieHA}, Coroallary 13 in~\cite{GG-Notes} and Section 2 of \cite{FrancisHH}.
Concretely, this claim arises from the fact that as a module over $A$ as an $E_n$ algebra, 
$M \in \Mod^{E_n}_A(\cC)$ is equipped with a commuting family of actions by $A$ parametrized by $\RR^n \setminus \{0\}\cong S^{n-1} \times \RR_{>0}$.
This action factors through the left action of the $E_1$ algebra $\int_{S^{n-1} \times \RR} A$ arising from factorization homology over the complement of the origin,
which integrates all those actions and through which they all factor. \mahmoud{i think the picture of why it is a one-sided action is clearer if we use $S^{n-1} \times \RR^+$ istead of $S^{n-1} \times \RR$}
As shown in Theorem 7.5.3.1 of \cite{LurieHA} and earlier in Proposition 3.16 of \cite{FrancisHH}, 
this algebra $\int_{S^{n-1} \times \RR} A$ is equivalent to ${\rm Free}(1_\cC)$, 
the image of the unit object $1_\cC \in \cC$ under the left adjoint ${\rm Free}$ to the forgetful functor $\Mod^{E_n}_A(\cC) \to \cC$.

We now use these results to prove the central claim of Theorem~\ref{thm: centralizer}.

\section{Proof of Theorem \ref{thm: centralizer}}

Let $f: \UU_n \fg \to B$ denote a map of $E_n$ algebras between the enveloping $E_n$ algebra of a dg Lie algebra $\fg$ and an $E_n$ algebra $B$.
The results of the preceding section imply
\begin{align*}
{\frak Z}_{E_n}(f) &\simeq {{\rm Mor}}_{\Mod^{E_n}_{\UU_n\fg}}(\UU_n \fg, B_f) \\
&\simeq {\rm Mor}_{\LMod_{\int_{S^{n-1}} \UU_n\fg}}(\UU_n \fg, B_f)
\end{align*}
where 
\begin{itemize}
\item[-] $B_f$ denotes the object in $\Mod^{E_n}_{\UU_n\fg}$, as well as in $\LMod_{\int_{S^{n-1}}\UU_n}$, determined by~$f$, 
\item[-] ${\rm Mor}_{\Mod^{E_n}_{\UU_n\fg}}(\UU_n \fg, B)$ denotes the classifying object for maps from $\UU_n\fg$ to $B_f$ in the $\infty$-category $\Mod^{E_n}_{\UU_n\fg}$ (i.e. an internal mapping object), and
\item[-] ${\rm Mor}_{\LMod_{\int_{S^{n-1}} \UU_n\fg}} (\UU_n \fg, B_f)$ denotes the classifying object for maps from ${\UU_n\fg}$ to $B$ in the $\infty$-category $\LMod_{\int_{S^{n-1}}\UU_n\fg}$.
\end{itemize}
Note that $\int_{S^{n-1}}\UU_n\fg$ is an $E_1$ algebra equivalent to the dg algebra $U(\fg \otimes k[\epsilon_{n-1}])$.
This is because the rational $(n-1)$-sphere is formal, and hence its cochains are quasi-isomorphic to its cohomology $k[\epsilon_{n-1}]$.
Using the canonical inclusion $i: U\fg \to U(\fg \otimes k[\epsilon_{n-1}])$,
we find
\[
k \otimes_{U\fg} U(\fg \otimes k[\epsilon_{n-1}]) \simeq \Sym(\fg[1-n]).
\]
Recall now that $\UU_n \fg \simeq \Sym(\fg[1-n])$ as $\fg$-modules \greg{shall we also need that it is true as $E_n$-algebra one we see right hand side as a $P_n$-algebra in the natural way and use formality ? I feel we need to say something more to recover the $E_n$-structure below ... something like the $E_n$-structure on Lie algebra cochains realizes the module map structure after all equivalences we use}.
Thus, base change \mahmoud{which is a map of $E_n$ algebras?} gives rise to an equivalence 
\[
{\rm Mor}_{\LMod_{\int_{S^{n-1}} \UU_n}} (\UU_n \fg, B_f)  \simeq {\rm Mor}_{U\fg} (k, B_f)
\]
of underlying chain complexes, i.e. an equivalence after applying forgetful functor. The right hand side is precisely Lie algebra cochains of $\fg$ with coefficients in~$B_f$, as desired.

\brian{do you like this example?} \greg{I do !}
\begin{eg}\label{eg: TT}
Consider the case $n=1$ and $B = U \fg$ where $f$ is simply the identity map. 
The Hochschild cohomology of $U \fg$ can be expressed in terms of Lie algebra cohomology as $\Hoch^*(U\fg,U\fg) \simeq \clies(\fg, U\fg)$. 
On the other hand, our model for the derived center ${\frak Z}(U(\fg))$ of $U \fg$ is the one-dimensional factorization algebra assigning 
\[
\clies(\fg \otimes \Omega^*(U), \cliels( \fg \otimes \Omega^*_c(U)))
\]
to each open set $U \subset \RR$.
In particular, the factorization homology of the derived center along the closed manifold $S^1$ can be computed as
\[
\int_{S^1} {\frak Z}(U(\fg)) \simeq \clies(\fg[\epsilon], \cliels( \fg[\epsilon])),
\]
where $\epsilon$ is a formal variable of degree~$1$. 
\brian{We should compare with Tamarkin-Tsygan's noncommutative calculus (cf. section 2.6).
We could exhibit Theorem 2.7.1 explicitly in this case for the $E_n$ case with enveloping algebras.}
\end{eg}

\section{A variant of the Loday-Quillen-Tsygan theorem}

Throughout this section, let $A$ be a dg algebra over a field $k$ of characteristic zero. We do not assume $A$ is unital, unless we state otherwise. More generally, it possible to work with $A_\infty$ algebras, which changes nothing essentially, while it adds a distracting layer of complexity. Let $\cL(A)$ stand for the dg Lie algebra $\gl_\infty(A)$ of infinite matrices with only finitely many nonzero entries in~$A$, and $\cC(A)$ the cochain complex $\Cyc_*(A)[1]$ thought of as an abelian differential graded algebra. Let $\gl_\infty$ stand for $\gl_\infty(k)$ and $\sl_\infty=\sl_\infty(k)$ denote the sub Lie algebra of those elements that have trace zero. There is a natural action of $\sl_\infty$ on $\gl_\infty(A)$, and hence on its Chevalley-Eilenberg chains $\cliels(\cL(A))$. A key step in the LQT theorem is the construction of an isomorphism of graded vector spaces
\beqn\label{lqtco}
\lambda^{co}_A: \cliels(\cL(A))_{\sl_\infty} \xto{\cong} \Sym(\cC(A)).
\eeqn
When $A$ is unital, the inclusion of the identity $k \hookrightarrow A$ means the action of $\sl_\infty$ on $\gl_\infty(A)$ is inner, 
and hence the projection $\cliels(\cL(A)) \to \cliels(\cL(A))_{\sl_\infty}$ is a quasi-isomorphism.
The standard statement of the LQT theorem composes these two maps.

\begin{thm}[\cite{LQ}]
For $A$ a unital dg algebra over a field $k$ of characteristic zero, there is a quasi-isomorphism
\beqn\label{lqt}
\lambda_A: \cliels(\cL(A)) \xto{\simeq} \Sym(\cC(A))
\eeqn
of dg cocommutative Hopf algebras.
\end{thm}

While the full Hopf algebra structure is typically emphasized, it is the coalgebra structure that is of particularly importance to us.

The naturality of the above constructions in $A$, leads to the following more abstract yet stronger statement.

\begin{cor}
There is a natural transformation
\[
\lambda: \cliels(\cL(-)) \Longrightarrow \Sym(\cC(-))
\]
between these two functors from the dg category of unital dg algebras to the dg category of dg cocommutative Hopf algebras over~$k$. Furthermore, the induced natural transformation on the homotopy categories is a natural isomorphism. 
\end{cor}
\greg{why not state (in a remark) this corollary and the next one as saying this is an equivalence in the $\infty$-categories of functor ?}
For our purposes, the following variant, which follows easy by formal arguments, is more useful.

\def\lqt{{\bblambda}}

\begin{thm}
\label{thm: variant LQT}
There is a natural transformation
\[
\lqt: \clies(\sl_\infty, \cliels(\cL(-)))  \Longrightarrow \clies(\sl_\infty) \otimes \Sym(\cC(-))
\]
between these two functors from the dg category of {\em non-unital} dg algebras to the dg category of dg Hopf algebras over~$k$.
For each nonunital dg algebra $A$, the map
\beqn\label{lqtinv}
\lqt_A: \clies(\sl_\infty, \cliels(\cL(A))) \xto{\simeq} \clies(\sl_\infty) \otimes \Sym(\cC(A))
\eeqn
is a quasi-isomorphism.
\end{thm}

This result yields the central claim of Theorem~\ref{thm: large N} about the large $N$ limit of the centralizer.

\begin{rmk}
Loday and Procesi proved versions of the LQT theorem for the Lie algebras $\mathfrak{o}_n$ and $\mathfrak{sp}_{2n}$, in which cyclic homology is replaced by dihedral homology.
As nothing substantive changes in proving analogous versions of our results above, we do not spell out the details here.
\end{rmk}

\begin{proof}
We take for granted that the reader is familiar with the proof of the LQT theorem as articulated in \cite{LQ, Weibel, LodayCyclic}.
Our primary modification is to replace the coinvariants $\cliels(\cL(A))_{\sl_\infty}$ by the invariants $\cliels(\cL(A))^{\sl_\infty}$.
Note that there is a canonical map
\beqn\label{invtoco}
\cliels(\cL(A))^{\sl_\infty} \hookrightarrow \cliels(\cL(A)) \to \cliels(\cL(A))_{\sl_\infty},
\eeqn
which we now show is an isomorphism.

To see this, we need to understand how $\sl_\infty$ acts on the underlying graded module of $\cliels(\cL(A))$. Note that as a $k$-module, $\cL(A) \cong \gl_\infty \otimes A$, which is a free $k$-module. 
Each $n$th tensor power thus factors as
\[
(\cL(A)[1])^{\otimes n} \cong (\gl_\infty[1])^{\otimes n} \otimes A^{\otimes n}.
\]
Taking the coinvariants under the action of the symmetric group $S_n$, which acts diagonally on the $\gl_\infty$ factor and the $A$ factor, 
we obtain
\[
\Sym^n (\gl_\infty[1]) \otimes A.
\]
But the action of $\sl_\infty$ is then solely on the left hand factor, leaving $A$ untouched.

There is a natural decomposition of $\Sym^n (\gl_\infty[1])$ into a direct sum $T_n \oplus T_n^\perp$ of representations of $\sl_\infty$, where $T_n$ is trivial and the complement $T_n^\perp$ has no trivial components.
If we replace $\gl_\infty$ with the finite-dimensional Lie algebra $\gl_m$, 
the analogous decomposition of $\Sym^n (\gl_m[1])$ exists by semisimplicity. 
In fact, with $m > n$, invariant theory implies
\[
(\gl_m^{\otimes n})^{\sl_m} = (\gl_m^{\otimes n})_{\sl_m} \cong k S_n,
\]  
the group algebra of the symmetric group $S_n$.
A stabilization phenomenon then implies the decomposition as $m \to \infty$.
We note that we see here the equivalence of the $\sl_m$-invariants with the $\sl_m$-coinvariants,
which holds for every degree $n$ and hence yields the desired isomorphism~(\ref{invtoco}).
\owen{MZ points out we should show the filtered colimit here commutes with taking invariants (which is a limit), 
which should be true since filtered colimits commute with finite limits.}

We thus conclude that there is a decomposition
\[
\cliels(\cL(A)) = T(A) \oplus T^\perp(A)
\]
into a direct sum of cochain complexes,
where the first summand is a trivial $\sl_\infty$-module and the second has no trivial components.
We will not unravel explicitly how to construct $T(A)$ from the algebra $A$ and the modules~$T_n$,
but the reader should recognize that $T(A) = \cliels(\cL(A))^{\sl_\infty}$.

We now recall two familiar facts about the Lie algebra cohomology of a semisimple Lie algebra~$\fg$:
\begin{itemize}
\item the Lie algebra cohomology of an irreducible, nontrivial, finite-dimensional module $M$ is zero, and
\item for any trivial module $N$, there is a natural isomorphism $H^*_\Lie( \fg,N) \cong H^*_\Lie(\fg) \otimes N$.
\end{itemize}
Together these imply that for an arbitrary finite-dimensional module~$M$,
\[
\clies(\fg,M) \simeq \clies(\fg) \otimes M^\fg.
\]
Thus, we have a decomposition
\[
\clies(\sl_\infty,\cliels(\cL(A))) = \clies(\sl_\infty,T(A)) \oplus \clies(\sl_\infty,T^\perp(A)), 
\]
and there is a split surjection
\[
\bbsigma_A: \clies(\sl_\infty,\cliels(\cL(A))) \to \clies(\sl_\infty,T(A)) \cong \clies(\sl_\infty) \otimes T(A) 
\]
that is a quasi-isomorphism.

We define our desired map $\lqt$ as the composite
\[
\lqt_A = \left(\id_{\clies(\sl_\infty)} \otimes \lambda^{co}_A \right) \circ \bbsigma_A,
\]
which is a quasi-isomorphism by construction.

Our arguments, like those for the LQT theorem, are natural in the algebra~$A$ and hence we have constructed the natural transformation $\lqt$ implicitly.
\end{proof}

\section{A factorization algebra arising from cyclic homology}

We explain here how to construct a factorization algebra using cyclic homology,
much as Lie algebra homology leads to an explicit factorization algebra modeling the enveloping $E_n$ algebra. 
Throughout this section, let $k$ denote the field $\RR$ or $\CC$; 
by mimicking \cite{Knudsen}, one could also produce an explicit model over~$\QQ$.

\begin{dfn}
Let $A$ be a dg algebra over $k$, and let $\cA = A \otimes \Omega^*_c$ denote the associated cosheaf of dg algebras on~$\RR^n$.
Let $\cC_A$ denote the precosheaf $\Cyc_*(\cA)[1]$ of cochain complexes on $\RR^n$,
which assigns to each open set $U$, the 1-shifted cyclic chains on $\cA(U)$.
\end{dfn}
\greg{Here we shall either recall our model for cyclic chains, or first define the $\infty$-categorical universal statement and then precise a nice model. By the universal model, I mean define cyclic as the homotopy $S^1$-fixed point for the action on Hochschild chains. This will make the construction more transparent ?}
\mahmoud{cyclic is the homotopy quotient and negative cyclic the homotopy fixed points}
Note that for two algebras $B$ and $B'$,
\[
\Cyc_*(B) \oplus \Cyc_*(B') \simeq \Cyc_*(B \times B') .
\] 
The corresponding statement for ordinary Hochschild homology is induced by the projections $B \leftarrow B \times B' \to B'$, and any map of algebras that induces a quasi-isomorphism on Hochschild homology induces one on cyclic homology as well.
(See, for instance, \S2.2.12 of \cite{LodayCyclic}.)
Hence, for any finite collection of disjoint opens $U_1$, \dots, $U_k$, 
we have
\[
\cC_A(U_1 \sqcup \cdots \sqcup U_k) \simeq \cC_A(U_1) \oplus \cdots \oplus \cC_A(U_k).
\]
Moreover, for any nested sequence of inclusions $V_1 \subset V_2 \subset \cdots$ with $V = \bigcup_i V_i$,
there is a quasi-isomorphism
\[
{\rm hocolim}(\cC_A(V_1) \to \cC_A(V_2) \to \cdots) \simeq \cC_A(V).
\]
(See E.2.1.1 of \cite{LodayCyclic}.)
In this sense $\cC_A$ behaves like a cosheaf, up to quasi-isomorphism. 
This property underlies the following claim.

\begin{lmm}
\label{lem: cycfact}
The functor $\Sym(\cC_A)$ sending an open set $U \subset \RR^n$ to $\Sym(\cC_A(U))$ is a locally constant prefactorization algebra.
\end{lmm}

%\begin{rmk}
%We note that, as with the definition of the Chevalley-Eilenberg chains of a local Lie algebra,
%we use here a construction of cyclic chains that plays nicely with the kind of vector spaces relevant to this situation,
%namely smooth sections of vector bundles.
%Where the cyclic quotient $A^{\otimes n}/C_n$ would appear for an ordinary algebra in complex vector spaces,
%we take the $\Omega^{0,*}(X^n)/C_n$ and so on.
%\owen{I need to check that the $\Sym$ doesn't lead to issues \dots If we must, we can ignore the quasi-isomorphism and focus on the map just to cyclic homology.}
%\end{rmk}

\begin{proof}
It remains to provide the structure maps of the putative prefactorization algebra.
We need to provide a multilinear structure map 
\beqn
\label{eqn: desiredmap}
\cC(U_1) \times \cdots \times \cC(U_n) \to \cC(V)
\eeqn
for any finite set of pairwise disjoint opens $U_1,\ldots, U_n$ inside a larger open $V$.
For the cosheaf $\Omega^{0,*}_c$ on pairwise disjoint opens $U_1,\ldots, U_n$,
the isomorphism of dg algebras
\[
\Omega^{*}_c(U_1) \oplus \cdots \oplus \Omega^{*}_c(U_n) \cong \Omega^{*}_c(U_1 \sqcup \cdots \sqcup U_n),
\]
determines a quasi-isomorphism
\beqn
\label{eqn:cyccosheaf}
\Cyc_*(A \otimes \Omega^{*}_c(U_1)) \oplus \cdots \oplus \Cyc_*(A \otimes \Omega^{*}_c(U_n)) \xto{\simeq} \Cyc_*(A \otimes \Omega^{*}_c(U_1 \sqcup \cdots \sqcup U_n)).
\eeqn
The inclusion $U_1 \sqcup \cdots \sqcup U_n \hookrightarrow V$ then provides a map
\[
\Cyc_*(A \otimes \Omega^{*}_c(U_1 \sqcup \cdots \sqcup U_n)) \to \Cyc_*(A \otimes \Omega^{*}(V)),
\]
and so determines a map
\beqn
\label{eqn:map2}
\Sym(\cC_A(U_1 \sqcup \cdots \sqcup U_n)) \to \Sym(\cC_A(V)).
\eeqn
Likewise, applying $\Sym$ to map~\eqref{eqn:cyccosheaf} provides
\[
\Sym(\cC_A(U_1)) \otimes \cdots \otimes \Sym(\cC_A(U_n)) \to \Sym(\cC_A(U_1 \sqcup \cdots \sqcup U_n)).
\]
We thus obtain the desired map \eqref{eqn: desiredmap} as a composite.
This construction is automatically associative for nested inclusions of pairwise disjoint opens,
and so $\Sym(\cC_A)$ is a prefactorization algebra.
\end{proof}

\begin{cor}
For $A$ a dg algebra over $k$, Theorem~\ref{thm: variant LQT} provides a weak equivalence of prefactorization algebras
\[
\lqt: \clies(\sl_\infty(\Omega^*), \cliels(\cL(\cA))) \xto{\simeq} \clies(\sl_\infty(\Omega^*)) \otimes \Sym(\cC_A).
\]
Hence, the functor $\Sym(\cC_A)$ determines a locally constant factorization algebra on $\RR^n$, and hence an $E_n$ algebra.
\end{cor}

This result yields the latter part of Theorem~\ref{thm: large N}, about an explicit factorization algebra.

\begin{rmk}
The Loday and Procesi theorem for the Lie algebras $\mathfrak{o}_n$ and $\mathfrak{sp}_{2n}$ replaces cyclic homology with the dihedral homology and leads to an analogous class of factorization algebras.
\end{rmk}


\section{\owen{A stub about a twisted version of LQT}}

\owen{Brian and I will later insert an expansion of the following.}

Let $\chi \in \Cyc^m(A)$ is a cyclic cocycle of degree $m$. 
Then $\chi$ determines a central extension $\cC(A)_\chi$ of $\Cyc_*(A)[1]$ as an abelian Lie algebra (cf. Heisenberg Lie algebras),
but $\chi$ also maps to Lie algebra cochains $\clies(\gl_N(A))$ for every $N$ via the LQT map,
and hence determines a central extension $\gl_N(A)_\chi$ of $\gl_N(A)$ as well.
There is thus a $\chi$-twisted version of the LQT map
\[
\lqt_\chi: \Sym(\cC(A)_\chi) \to \cliels(\gl_N(A)_\chi)
\]
which is a quasi-isomorphism in the large $N$ limit.

This induces $\chi$-twisted versions of all our enveloping $E_n$ algebra theorems (in large $N$ limit).

%We briefly remark on a natural modification of the prefactorization algebra $\Sym(\cC_A)$. 
%For simplicity, suppose $A$ is an ordinary (non dg) algebra, and let $\chi \in \Cyc^m(A)$ be a cyclic {\em cocycle} of degree $m$.
%Consider for each open $U \subset \RR^n$ the cochain complex
%\[
%\Tilde{\cC}_{A,\chi} (U) = \left(\Cyc_*(A \tensor \Omega^*_c(U))[1] \oplus \CC \cdot K , \d_{\rm Cyc} + K \cdot \chi \right) 
%\]
%where $K$ is a formal parameter of degree $-m+n+1$.
%Also, $\d_{\rm Cyc}$ is the cyclic differential and $\chi$ is defined on $m$-chains by the pairing between cyclic chains and cochains, together with integration along $U$:
%\[
%\chi ((a_1 \otimes \alpha_1) \tensor \cdots \tensor (a_m \otimes \alpha_m)) = \int_U \<\chi, a_1 \tensor \cdots \tensor a_m\> \alpha_1 \wedge \cdots \wedge \alpha_m.
%\]
%The differential is extended to the entire cochain complex by the graded Leibniz rule. 
%By similar arguments as above, the assignment $U \subset \Tilde{\cC}_{A, \chi}$ is a precosheaf of cochain complexes and $\Sym\left(\Tilde{\cC}_{A,\chi}\right)$ is a prefactorization algebra. 
%As a precosheaf of graded vectors spaces one has $\Tilde{\cC}_{A, \chi} = \cC_A \oplus \ul{\CC}[m-n-1]$, where $\ul{\CC}$ is the constant precosheaf, so that we may think of $\Sym\left(\Tilde{\cC}_{A,\chi}\right)$ as a twisted version of $\Sym(\cC_A)$. 
%These types of twisted \brian{what to say? check my degrees!}
%



\bibliographystyle{alpha}  
\bibliography{centralize}

%\bibliographystyle{spmpsci}  

\end{document}
