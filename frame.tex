\documentclass[11pt]{amsart}

\usepackage{macros}
\usepackage[bbgreekl]{mathbbol}

\linespread{1.25}

%\usepackage[final]{pdfpages}

\setcounter{tocdepth}{2}
\numberwithin{equation}{section}



\def\brian{\textcolor{blue}{BW: }\textcolor{blue}}
\def\owen{\textcolor{magenta}{OG: }\textcolor{magenta}}

\begin{document}
\title{Large $N$ limits of higher enveloping algebras}

%\author{Owen Gwilliam}
%\address{Department of Mathematics and Statistics \\
%Lederle Graduate Research Tower, 1623D \\
%University of Massachusetts Amherst \\
%710 N. Pleasant Street}
%\email{gwilliam@math.umass.edu}
%
%\author{Brian Williams}
%\address{Department of Mathematics, 
%Northeastern University \\ 
%567 Lake Hall \\ 
%Boston, MA 02115 \\ U.S.A.}
%\email{br.williams@northeastern.edu}


\maketitle
\thispagestyle{empty}

\tableofcontents
 
Recently there has emerged a systematic generalization of the enveloping algebra $U\fg$ of a Lie algebra $\fg$ to {\em higher} enveloping algebras:
each dg Lie algebra $\fg$ has an enveloping $E_n$ algebra $\UU_n\fg$.
Building on prior work \cite{BD, AF}, Knudsen \cite{Knudsen} has developed these ideas extensively and offered useful models as factorization algebras.
Our central goal here is to develop close cousins of two important results about enveloping algebras of Lie algebras: 
\begin{enumerate}
\item[(1)] the (underived) center of $U\sl_{N+1}$ is isomorphic to $(\Sym(\CC^{N}))^{S_N}$, where we view $\CC^{N}$ as the Cartan subalgebra of $\sl_{N+1}$ and $S_N$ as the Weyl group of $SL_{N+1}$, and
\item[(2)] the derived center of the enveloping algebra $U\fg$ is given by derived invariants of the adjoint action of $\fg$ on $U\fg$:
\[
Hoch^*(U\fg,U\fg) \simeq \clies(\fg, Ug),
\]
\end{enumerate}
but in a way that works uniformly for these higher enveloping algebras.

Our results are cousins, and not exact analogues, because we are interested in varying (1) by taking the limit as $N \to \infty$ and in varying (2) by studying the derived {\em centralizer} of a map $\UU_n\sl_\infty \hookrightarrow \UU_n\gl_\infty$, where $A$ is a dg algebra.
(The derived center is the centralizer of the identity map $\UU_n\fg \to \UU_n\fg$.)
These notions will be reviewed and explained in the next section below,
where we also describe our results in detail.
When $n=1$, however, note that we find that \owen{state the classical thing!}

Our main tool is the Loday-Quillen-Tsygan (LQT) theorem,
which describes the large $N$ limit of Lie algebra homology of $\gl_N(A)$ for $A$ a dg algebra.
Experts will know that enveloping $E_n$ algebras can be constructed using Chevalley-Eilenberg chains
(i.e., that they are shifted versions of $\cliels$, in a precise sense),
and hence will recognize the relevance of the LQT theorem in this context.
From this perspective, our main result can be understood as an $E_n$ generalization of the LQT theorem.

We hope these constructions and ideas have interesting applications to other large $N$ phenomena,
such as in quantum field theory and random matrix theory.

\section{Our main results}

Our primary result is the following.

\begin{thm}
Let $A$ be a unital dg algebra over a field $k$ of characteristic zero. 
Then the centralizer of the Lie algebra map 
\[
u_A: \sl_\infty(k) \hookrightarrow \gl_\infty(A)
\]
\end{thm}

\section{A variant of the Loday-Quillen-Tsygan theorem}

Throughout this section, let $A$ be a dg algebra over a field $k$ of characteristic zero.
We do not assume $A$ is unital; whenever unitality leads to a stronger result, we will explicitly add it as a hypothesis.
(It is possible to work with $A_\infty$ algebras rather than dg algebras, 
but it changes nothing structurally essential while adding a distracting layer of notational complexity.)

\begin{dfn}
Let $\cC(A)$ denote the cochain complex $\Cyc_*(A)[1]$, which may be viewed as an abelian Lie algebra. 
Let $\cL(A)$ denote the dg Lie algebra $\gl_\infty(A)$.
\end{dfn}

Let $\sl_\infty$ denote $\sl_\infty(k)$, the trace-zero matrices of rank infinity with only finitely many nonzero entries in~$k$.
Let $\gl_\infty$ denote $\gl_\infty(k)$, the matrices of rank infinity with only finitely many nonzero entries in~$k$.
Let $\gl_\infty(A)$ denote the matrices of rank infinity with only finitely many nonzero entries in~$A$.
There is a natural action of $\sl_\infty$ on $\gl_\infty(A)$, 
and hence on its Chevalley-Eilenberg chains $\cliels(\cL(A))$.
A key step in the LQT theorem is the construction of a quasi-isomorphism
\beqn\label{lqtco}
\lambda^{co}_A: \cliels(\cL(A))_{\sl_\infty} \xto{\cong} \Sym(\cC(A)).
\eeqn
When $A$ is unital, the inclusion of the identity $k \hookrightarrow A$ means the action of $\sl_\infty$ on $\gl_\infty(A)$ is inner, 
and hence the projection $\cliels(\cL(A)) \to \cliels(\cL(A))_{\sl_\infty}$ is a quasi-isomorphism.
The standard statement of the LQT theorem composes these two maps.

\begin{thm}
For $A$ a unital dg algebra over a characteristic zero field $k$,
there is a quasi-isomorphism
\beqn\label{lqt}
\lambda_A: \cliels(\cL(A)) \xto{\simeq} \Sym(\cC(A))
\eeqn
of dg cocommutative Hopf algebras.
\end{thm}

For us it is the coalgebra structure that is particularly important, although the full Hopf algebra structure is typically emphasized.

As the constructions are natural in the algebra $A$, one can phrase a stronger but more abstract statement.

\begin{cor}
There is a natural transformation
\[
\lambda: \cliels(\cL(-)) \Longrightarrow \Sym(\cC(-))
\]
between these two functors from the dg category of unital dg algebras to the dg category of dg cocommutative Hopf algebras over~$k$.
It is a quasi-isomorphism on every algebra.
\end{cor}

For our purposes, a variant is more useful, but this variant follows by easy, formal arguments.

\def\lqt{{\bblambda}}

\begin{thm}
There is a natural transformation
\[
\lqt: \clies(\sl_\infty, \cliels(\cL(-)))  \Longrightarrow \clies(\sl_\infty) \otimes \Sym(\cC(-))
\]
between these two functors from the dg category of {\em non}unital dg algebras to the dg category of dg Hopf algebras over~$k$.
For each nonunital dg algebra $A$, the map
\beqn\label{lqtinv}
\lqt_A: \clies(\sl_\infty, \cliels(\cL(A))) \xto{\simeq} \clies(\sl_\infty) \otimes \Sym(\cC(A))
\eeqn
is a quasi-isomorphism.
\end{thm}

\begin{proof}
We take for granted that the reader is familiar with the proof of the LQT theorem as articulated in \cite{LQ, Weibel, Loday}.
Our primary modification is to replace the coinvariants $\cliels(\cL(A))_{\sl_\infty}$ by the invariants $\cliels(\cL(A))^{\sl_\infty}$.
Note that there is a canonical map
\beqn\label{invtoco}
\cliels(\cL(A))^{\sl_\infty} \hookrightarrow \cliels(\cL(A)) \to \cliels(\cL(A))_{\sl_\infty},
\eeqn
and in fact this map is an isomorphism, which we now show.

To see this claim, we need to understand how $\sl_\infty$ acts on the underlying graded module of $\cliels(\cL(A))$. Note that as a $k$-module, $\cL(A) \cong \gl_\infty \otimes A$, which is a free $k$-module. 
Each $n$th tensor power thus factors as
\[
(\cL(A)[1])^{\otimes n} \cong (\gl_\infty[1])^{\otimes n} \otimes A^{\otimes n}.
\]
Taking the coinvariants under the action of the symmetric group $S_n$
(which acts diagonally on the $\gl_\infty$ factor and the $A$ factor),
we obtain
\[
\Sym^n (\gl_\infty[1]) \otimes A.
\]
But the action of $\sl_\infty$ is then wholly on the left hand factor, leaving $A$ independent.

There is a natural decomposition of $\Sym^n (\gl_\infty[1])$ into a direct sum $T_n \oplus T_n^\perp$ of representations of $\sl_\infty$, where $T_n$ is trivial and the complement $T_n^\perp$ has no trivial components.
If we replace $\gl_\infty$ with the finite-dimensional Lie algebra $\gl_m$, 
the analogous decomposition of $\Sym^n (\gl_m[1])$ exists by semisimplicity. 
In fact, with $m > n$, invariant theory implies
\[
(\gl_m^{\otimes n})^{\sl_m} = (\gl_m^{\otimes n})_{\sl_m} \cong k S_n,
\]  
the group algebra of the symmetric group $S_n$.
A stabilization phenomenon then implies the decomposition as $m \to \infty$.
We note that we see here the equivalence of the $\sl_m$-invariants with the $\sl_m$-coinvariants,
which holds for every degree $n$ and hence yields the desired isomorphism~(\ref{invtoco}).

We thus conclude that there is a decomposition
\[
\cliels(\cL(A)) = T(A) \oplus T^\perp(A)
\]
into a direct sum of cochain complexes,
where the first summand is a trivial $\sl_\infty$-module and the second has no trivial components.
We will not unravel explicitly how to construct $T(A)$ from the algebra $A$ and the modules~$T_n$,
but the reader should recognize that $T(A) = \cliels(\cL(A))^{\sl_\infty}$.

We now recall two familiars fact about the Lie algebra cohomology of a semisimple Lie algebra~$\fg$:
\begin{itemize}
\item the Lie algebra cohomology of an irreducible, nontrivial, finite-dimensional module $M$ is zero, and
\item for any trivial module $N$, there is a natural isomorphism $H^*_\Lie( \fg,N) \cong H^*_\Lie(\fg) \otimes N$.
\end{itemize}
Thus, we have a decomposition
\[
\clies(\sl_\infty,\cliels(\cL(A))) = \clies(\sl_\infty,T(A)) \oplus \clies(\sl_\infty,T^\perp(A)), 
\]
and there is a split surjection
\[
\bbsigma_A: \clies(\sl_\infty,\cliels(\cL(A))) \to \clies(\sl_\infty,T(A)) \cong \clies(\sl_\infty) \otimes T(A) 
\]
that is a quasi-isomorphism.

We define our desired map $\lqt$ as the composite
\[
\lqt_A = \left(\id_{\clies(\sl_\infty)} \otimes \lambda^{co}_A \right) \circ \bbsigma_A,
\]
which is a quasi-isomorphism by construction.

Our arguments, like those for the LQT theorem, are natural in the algebra~$A$ and hence we have constructed the natural transformation $\lqt$ implicitly.
\end{proof}

%\bibliographystyle{alpha}
%%\bibliographystyle{spmpsci}  
%\bibliography{hic}

\end{document}